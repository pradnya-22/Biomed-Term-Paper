\documentclass[12pt]{article}
\usepackage[english]{babel}
\usepackage{natbib}
\usepackage{url}
\usepackage[utf8x]{inputenc}
\usepackage{amsmath}
\usepackage{graphicx}
\graphicspath{{images/}}
\usepackage{parskip}
\usepackage{fancyhdr}
\usepackage{vmargin}
\setmarginsrb{3 cm}{2.5 cm}{3 cm}{2.5 cm}{1 cm}{1.5 cm}{1 cm}{1.5 cm}

\title{LASER SURGERY}								
\author{21111038}								
\date{01 April 2022}											% Date

\makeatletter
\let\thetitle\@title
\let\theauthor\@author
\let\thedate\@date
\makeatother

\pagestyle{fancy}
\fancyhf{}
\rhead{\theauthor}
\lhead{\thetitle}
\cfoot{\thepage}

\begin{document}

%%%%%%%%%%%%%%%%%%%%%%%%%%%%%%%%%%%%%%%%%%%%%%%%%%%%%%%%%%%%%%%%%%%%%%%%%%%%%%%%%%%%%%%%%

\begin{titlepage}
	\centering
    \vspace*{0.5 cm}
    \includegraphics[scale = 0.17]{logo.jpg}\\[1.0 cm]
    \textsc{\LARGE  National Institute of Technology\newline\newline Raipur}\\[2.0 cm]	
	\textsc{\Large PROJECT REPORT }\\[0.5 cm]			
	\rule{\linewidth}{0.2 mm} \\[0.4 cm]
	{ \huge \bfseries \thetitle}\\
	\rule{\linewidth}{0.2 mm} \\[1.5 cm]
	
\begin{minipage}{0.4\textwidth}
	\begin{flushleft} \large
		{\textbf{Submitted by:}}\\
		Pradnya Manmode\\
        21111038\\
        First Semester, B.tech\\
        Biomedical Engineering,\\
        National Iinstitute of Technology, Raipur.\\
			\end{flushleft}
			\end{minipage}~
			\begin{minipage}{0.4\textwidth}
            
	\end{minipage}\\[2 cm]
	
	
    
    
    
    
	
\end{titlepage}

\newpage
\pagestyle{fancy}
\begin{center}
\textsc{\huge\textbf{Acknowledgement}}   
\end{center}

\indent
\large

I would like to extend my sincere thanks to my teacher I would like to extend my sincere thanks to my teachers Dr. Saurabh Gupta Sir, Sumit Banchhor Sir, Arindam Bit Sir, Neelam Shobha Mam, who has helped me in this endeavour and has always been very cooperative, gave me valuable suggestions and guidelines during the completion of the project. Without their help and guidance, the project on the topic{\textbf{“Laser Surgery”}} wouldn’t be as good as this. 

\indent

\begin{flushright}
{\textbf{Pradnya Manmode}}, \\
\textbf{21111038} \\
\textbf{First Semester} \\
\textbf{Biomedical Engineering} \\
\textbf{National Institute of Technology, Raipur.} \\ 
\end{flushright}

\indent 

\textbf{Date of submission: 6th April 2022}

\newpage
\pagestyle{fancy}
\begin{center}
    \textsc{\huge\textbf{ABSTRACT}}
\end{center}
 
 \indent
 

\hspace{3cm}Lasers are used in the operating room for a variety of surgical procedures. Surgeons utilize lasers to cut, coagulate and remove tissue in general surgery and surgical specialties. Because of its minimum intrusive effect, laser therapy has become a popular topic in modern medicine. \\
Nowadays, Lasers are widely employed in the treatment and diagnosis of a wide range of disorders, including malignancies, lithotripsy, ophthalmology, dermatology, and cosmetic operations. Most lasers have replaced conventional surgical instruments for superior wound healing results, depending on the type of laser, wavelength, and delivery mechanism. 
New lasers have been developed over time employing a variety of equipment and technologies, and they are now used in a wide range of medical specialties.\\
Laser applications in surgery are described in this report, as well as their benefits compared to earlier procedures, with the goal of delivering appropriate therapeutic and non-invasive treatments with minimum post-surgical adverse effects.
 
\newpage
\pagestyle{fancy}
\begin{center}
    \textsc{\huge\textbf{Introduction}}
\end{center}
 
 \indent
 
\hspace{2.5cm}Laser surgery is a type of surgery that uses the cutting power of a laser beam to achieve bloodless tissue cuts or to remove a superficial lesion such as a skin tumour.

\hspace{2.5cm} Minimally invasive surgery is gaining popularity these days. In some fields of surgery, procedures have been perfected to the point where no or minor incisions are needed. There are many distinct types of lasers, each with its own wavelengths of emitted light, power, and capacity to coagulate, cut, or evaporate tissue.
\indent

\hspace{2.5cm}Lasers are used to treat disorders that cause bleeding or blockage. Colon polyps and tumours that cause intestinal or gastric blockage are treated using lasers to reduce, kill, and eradicate them. Laser therapy is sometimes used alone, although it is frequently used in conjunction with surgery, chemotherapy, or radiation therapy. Lasers can shut nerve endings to minimize postoperative pain and seal lymph veins to reduce swelling and tumour cell development.

\newpage

\hspace{2.5cm}Lasers are utilized in a variety of surgical and non-contact surgical procedures. Laser radiation heats a particular surgical tip in the first case, which is then employed for tissue excision via thermal conduction. This technique of operation works best with diode lasers. In a non-contact mode, on the other hand, the laser source wavelength is chosen to use a large quantity of water in most soft tissues. This operation approach is compatible with thulium-doped fiber lasers, erbium-doped fiber lasers, and Mid-IR hybrid lasers.\\
\indent 
\hspace{2cm}In several disciplines of medicine, including as cancer treatment and tumour ablation, brain operations, epilepsy, cardiology, atrial fibrillation, lithotripsy, dermatology, skin rejuvenation, and lipolysis, multiple laser systems are being researched (see Figure 1). The laser's energy is a safe and effective instrument for treating malignancies, ablation of aberrant conductive pathways, and numerous reconstructive and cosmetic operations. 7 This review is intended to provide a new viewpoint on the use of laser in surgery due to advancements in laser application and its advantageous aspects\\.
\indent
\hspace{2.5cm} There are various applications of laser surgery like Soft-tissue, Dermatology and plastic surgery, Eye surgery, Endovascular Surgery, Foot and ankle surgery, Gastrointestinal tract, Oral and dental surgery, Spine surgery, Thoracic surgery, Cardiovascular Surgery, Brain surgery, and many more.  Laser therapy is also used to treat Cancers and precancers on the surface of the body or the lining of internal organs.

\Large{Let’s have a look at a few of them.}

\section{Cancer Treatment and Tumour Surgery:}

\hspace{2.5cm}Lasers are now considered to be safe treatments for a variety of malignancies in diverse organ systems.
The use of laser ablation techniques to treat superficial gastrointestinal cancers like superficial esophageal and early gastric tumours, colorectal adenoma, and high-grade Barrett's oesophagus has been successful and widespread. For certain forms of lung cancer lesions, laser photodynamic therapy is an effective treatment.\\
\indent
\hspace{2cm} The photochemical and photothermal effects of direct laser ablation have been utilized to kill cancer cells directly. Photochemical reactions produce harmful radicals, which cause tissue death, stress, and fragmentation, as well as warmth, blood coagulation, and cell death. The photodynamic approach was created over a century ago to target tumour cells. This treatment involves the administration of a photosynthetic drug, followed by irradiation of the desired area with visible light proportional to the photosynthetic drug's absorption wavelength. The photosynthesizer generates reactive oxygen species, which are harmful to neoplastic cells, by first forming a single excited state and subsequently a triplet state. Selective photothermal therapy makes use of the best "light absorbed dyes" to boost laser-induced tumor cell death. \\

\section{Brain Surgery:}

\hspace{2.5cm} Malignant brain tumors are treated first with surgical removal, followed by chemotherapy and radiation therapy. Glioblastoma surgery, on the other hand, results in lasting neurologic impairments but improves survival over chemotherapy and radiation alone. Because of their various comorbidities, some patients are not surgical candidates. Furthermore, the existence of deep lesions, associated symptoms, low functional scores, and general anesthetic incapacity are all factors that impede surgical removal, limiting life.\\
\indent
\hspace{2cm}The efficiency and safety of laser interstitial thermal treatment (LITT) have both improved in recent years. LITT is a minimally invasive percutaneous therapy that uses light energy to target tissue through a fibropathic catheter, resulting in selective thermal ablation of malignant and benign lesions.\\
\indent
\hspace{1.8cm}
Real-time MRI creates a dynamic temperature map of the brain during operation. This enables continuous temperature monitoring of brain tissue and assures that laser-induced damage is efficiently localized on the tumor and limited to the healthy tissue surrounding it.\\
\indent
\hspace{2cm} LITT is a safe and effective alternative to high-grade malignant glioma and recurrent metastatic lesions in tumor surgery and chronic pain. Epilepsy, radiation necrosis, refractory cerebral edema, and tumors such as meningioma, ependymoma, primitive neuroectodermal tumor, chordoma, and hemangioblastoma are all treated with LITT. In comparison to other stereotactic techniques such as radiofrequency thermocoagulation, gamma knife, and targeted ultrasound, LITT produces low-risk invasive soft tissue pathological demolition. Furthermore, during deep lesions, damage to sections of the cerebral cortex can be reduced or completely avoided, and epileptic foci near scattered or even important areas of the brain can be treated. The procedure does not require a craniotomy and is considered minimally invasive because it just requires a minor incision and puncture to guide the laser fiber. Most patients are discharged the next day after the procedure, which usually takes 3-4 hours.\\

\section{Cardiovascular Surgery:}

\hspace{2.5cm} Treatment methods that improve and increase blood flow through the coronary arteries include angioplasty, coronary artery bypass surgery, and medicines. When all of these options have been exhausted, the patient has no choice but to undergo surgery, with the exception of a few rare cases of heart transplantation. In the absence of a feasible alternative to surgery, the patient is usually treated with therapeutic medicines, which are sometimes accompanied by considerable lifestyle limitations. The innovative methods of improving blood flow to the heart sections not treated by angioplasty or surgery include trans-myocardial laser revascularization (TMLR), laser vascular anastomosis, and laser angioplasty in peripheral arterial disorders. Laser uses in cardiovascular surgery are extremely uncommon around the world. \\
\indent
\hspace{2cm} In ischemic heart disease with areas that cannot be bypassed, TMLR is performed through a small split in the left chest between the ribs (thoracotomy) under general anesthesia with or without coronary bypass surgery. It is the sole treatment for severe angina and is used in conjunction with coronary artery bypass grafting (CABG). The CO2 laser or the Ho: YAG laser is used in TMLR to target specific parts of the heart muscle.\\
\indent
\hspace{2cm} In TMLR, angina relief is more important than maximal medical treatment for coronary artery disease that is not revascularized. Combining CABG with TMLR has led to the improvement of symptoms and no additional risk. Angiogenesis is a possible mechanism in which TMLR has benefited from interacting with tissue. After TMLR, perfusion and concomitant improvement in myocardial function have been observed.\\

\section{Oral Surgery:}

\hspace{2.5cm} The clinical use of lasers in dentistry and oral and facial cosmetic surgery has been expanded thanks to recent advancements in laser technology. In oral surgery, lasers such as CO2, Er: YAG, Diode, and Nd: YAG are commonly employed. Low-level lasers are also used to help with disinfection and healing treatments. A multitude of oral disorders is treated with laser technology, including oral mucosa, oral benign lesions, oral cavity cancer, and excisional biopsy. The treatment of patients with lesions of the oral mucosa is a therapeutic challenge. Laser therapy appears to be a useful alternative treatment for alleviating illness symptoms.\\

\section{Thoracic Surgery:}

\hspace{2.5cm} Surgical laser applications are most commonly employed in thoracic surgery to remove pulmonary metastases and cancers of various primary localizations. Surgical sectioning of the parenchyma, anatomic segmental resections, removal of tumors from the thoracic wall, and abrasion of the pleura parietals are some of the other applications. The number of possibly surgically resectable pulmonary nodules has increased dramatically with the introduction of surgical lasers. Other traditional surgical procedures, such as segmental or wedge resections with surgical stapling, will usually result in more lung tissue loss than laser surgery, especially in individuals with many pulmonary nodules.\\
\section{Laser lithotripsy:}

\hspace{2.5cm} Laser lithotripsy is a commonly accepted method for the fragmentation of urinary and biliary stones. Lasers with a photochemical device can perform lithotripsy and photothermal effect. Greenlight at 504 nm is produced and absorbed mostly by the yellow-colored urinary calculi. It can be safely used without damaging the surrounding tissues. This stone absorbs energy through the laser; excited ions can build up around the stone, creating a shock wave to break the stone into fragments. This laser is inefficient against colorless and non-absorbable compounds such as those consisting of cysteine, so photosynthetic sensors have been successfully used as irrigation and absorbent fluids to initiate the fragmentation process. The Q-switched Nd: YAG laser uses this mechanism to perform lithotripsy and generate a larger shock wave. The long-pulsed holmium: YAG laser, which makes the light at a 2100 nm wavelength with extreme absorption by water, mostly uses a photothermal process to break up calculi. The peripheral fluid is heated after the absorption of energy. Some created steam separates from the water and causes fragmentation.  Ho: YAG laser lithotripsy is the most effective endoscopic method with higher rates of stone fragmentation for treating ureteral stones compared to pneumatic lithotripsy. Also, it is safe and efficacious and works more desirable than other methods. Furthermore, it is used for the fragmentation of biliary stones.

\newpage
\pagestyle{fancy}
\begin{center}
    \textsc{\huge\textbf{CONCLUSION}}
\end{center}
 
 \indent

\Large

\hspace{2.5CM}Nowadays, laser surgery is widely used in medicine. This method is linked to reduced bleeding, a faster recovery time after surgery, and fewer adverse effects. The high cost and reliance on a specialist surgeon, as well as the application of a proper laser setup and the availability of standard laser devices, are all significant challenges with laser application in surgery.


\newpage

\textbf{REFERENCES:}
\\

 {https://en.wikipedia.org/wiki/Laser-surgery}\\

 { https://stanfordhealthcare.org/medical-treatments/l/laser/types/laser-surgery.html}\\

 {https://www.wikiwand.com/en/Laser-surgery}







\end{document}